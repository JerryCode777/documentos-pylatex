\documentclass{article}%
\usepackage[T1]{fontenc}%
\usepackage[utf8]{inputenc}%
\usepackage{lmodern}%
\usepackage{textcomp}%
\usepackage{lastpage}%
\usepackage{amsmath}%
%
\title{Uso de PyLaTeX para Generar Documentos LaTeX}%
\author{Jerry Anderson Huaynacho Mango}%
\date{\today}%
%
\begin{document}%
\normalsize%
\maketitle%
\section{Introducción}%
\label{sec:Introduccin}%
Este documento fue generado usando PyLaTeX, una librería de Python para crear archivos LaTeX programáticamente.%
 A continuación se muestran ejemplos de cómo trabajar con variables y tablas.

%
\section{Uso de Variables en PyLaTeX}%
\label{sec:UsodeVariablesenPyLaTeX}%
Definimos las variables x = 5 e y = 10.%
El resultado de la multiplicación de x por y es: %
\[%
5 \textbackslash{}times 10 = 50%
\]

%
\section{Generación de Tablas}%
\label{sec:GeneracindeTablas}%
La siguiente tabla muestra datos generados dinámicamente:%
\begin{tabular}{|c|c|c|}%
\hline%
ID&Nombre&Edad\\%
\hline%
1&Persona 1&25\\%
\hline%
2&Persona 2&30\\%
\hline%
3&Persona 3&35\\%
\hline%
4&Persona 4&40\\%
\hline%
5&Persona 5&45\\%
\hline%
\end{tabular}

%
\section{Conclusión}%
\label{sec:Conclusin}%
En este documento hemos mostrado cómo generar variables, realizar cálculos y crear tablas de manera dinámica usando PyLaTeX.

%
\end{document}